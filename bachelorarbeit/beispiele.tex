% Beispiel für Bildintegration
\begin{figure}[!ht]
	\centering
	\includegraphics[width=0.8\textwidth]{abbildungen/Logo.png}
	\caption[Beschreibung]{Beschreibung~\cite[S.14]{mf2005}}
       \label{fig:Beschreibung}
\end{figure}

% Beispiel für einen Schaltplan 

\begin{figure}
 \centering
\begin{circuitikz}
  \draw (0,80) node[vcc]    (vcc) {+ \SI{5}{\volt}};
  \draw (0,60) node (d) {};
  \draw (0,30) node (r3) {};
  \draw (0,15) node (r4) {};
  \draw (0,0)  node[rground] (gnd) {};
  \draw (vcc) to[leDo] (d.center);
  \draw (d.center)  to[R=$R_3$,a={\SI{1}{\kilo\ohm}}]  (r3.center);
  \draw (r3.center)  to[R=$R_4$,a={\SI{1}{\kilo\ohm}}]  (r4.center);
  \draw (r4.center)  to  (gnd);
\end{circuitikz}
\caption[Schaltung]{Schaltung}
  \label{fig:Schaltung}
\end{figure}

% Beispiel: Tabelle 
\begin{table}[!ht]
	\centering
	  \caption[Beispieltabelle]{Beispieltabelle~\cite[S.400]{KnutThea2009}}
	  \label{Beispieltabelle}
	  \begin{tabular}{ | l | c | }
	    \hline
	    Überschrift 1 & Überschrift 2 \\ \hline 
	    Info 1 & Info 2 \\ \hline
	    Info 3 & Info 4 \\ \hline
	    \hline
	    \multicolumn{2}{|c|}{Info in einer Zelle} \\
	    \hline
	  \end{tabular}
\end{table}

% Beispiel: Booktab
\begin{table}[!ht]
	\centering
	\caption[Beispieltabelle 2]{Beispieltabelle 2~\cite[S.700]{mf2005}}
	\label{Beispieltabelle_2}
      \begin{tabular}{ccc}\toprule
	A&B&C \\ \midrule
	a&b&c \\ \cmidrule{1-3}
	1&2&3\\ \bottomrule
	\end{tabular}
\end{table}

% Beispiel für eingerücktes Zitat (länger als drei Zeilen)
\begin{quote}
	\singlespacing \small
	"`Lorem ipsum dolor sit amet, consectetur adipiscing elit, sed do eiusmod tempor incididunt ut labore et dolore magna aliqua. 
	Et odio pellentesque diam volutpat commodo sed. Donec pretium vulputate sapien nec sagittis aliquam. Nullam ac tortor vitae 
	purus faucibus."'~\cite[S. 189]{KnutThea2009}
\end{quote}

%Beispiel für Formeln
\begin{quote}
	Die Funktion F: $\mathbb{R} \rightarrow$ [0,1] mit $F(t) = P (X \le t)$ heißt Verteilungsfunktion von $X$. vgl. \cite[S.55]{mf2005}
\end{quote}

% Beispiel für Listing
\lstset{language=xml}
\begin{lstlisting}[float=!ht, frame=htrbl, caption={die datei {\normalfont \ttfamily  data-config.xml} dient als beispiel für xml quellcode}, label={lst:dataconfigxml}]
<dataconfig>
  <datasource type="jdbcdatasource" 
              driver="com.mysql.jdbc.driver"
              url="jdbc:mysql://localhost/bms_db"
              user="root" 
              password=""/>
  <document>
    <entity name="id"
        query="select id, htmlbody, sentdate, sentfrom, subject, textbody
        from mail">
    <field column="id" name="id"/>
    <field column="htmlbody" name="text"/>
    <field column="sentdate" name="sentdate"/>
    <field column="sentfrom" name="sentfrom"/>
    <field column="subject"  name="subject"/>
    <field column="textbody" name="text"/>
    </entity>
  </document>
</dataconfig>
\end{lstlisting}

% Beispiel: Verweis auf eine Abbildung
%Abbildung~\ref{fig:Beschreibung} [S.\pageref{fig:Beschreibung}]

% Beispiel: Verweis auf eine Tabelle 
%Tabelle~\ref{table:Beispieltabelle} [S.\pageref{table:Beispieltabelle}]

% Beispiel Zitat
%~\autocite[S.55]{mf2005}

