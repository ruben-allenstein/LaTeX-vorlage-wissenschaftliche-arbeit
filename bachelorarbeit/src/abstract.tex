\section*{Abstract}\label{abstract}

Zunächst werden die wichtigsten Begriffe mit Bezug zu dem Thema Omni-Channel-Marketing erklärt und im weiteren Verlauf der Bachelorthesis die Merkmale des erfolgreichen E-Commerce mit Blick auf die Kosmetikbranche herausgearbeitet.
Von dem aktuellen Stand ausgehend werden die technologischen Fortschritte, Entwicklungen und Möglichkeiten für das Jahr 2023 identifiziert und analysiert.
\newline

Mit Unterstützung der Analyse wird eine Idee entwickelt, wie die Wettbewerbsfähigkeit eines Unternehmens für die Zukunft mit dem Einsatz einer Omni-Channel-Strategie verbessert werden kann. Die Fragestellungen werden mit Hilfe gesammelter Erfahrung durch interne Projektanwendungen, Fachliteratur und Studien diskutiert und beantwortet. Dabei kristallisiert sich heraus, dass die jüngeren Generationen ein anderes Konsumverhalten haben, als ältere Generationen, da sie mit dem Internet aufgewachsen sind. Durch die Technologien und die alltägliche Nutzung von Smartphones, Tablets und Computern wird der Online-Handel beeinflusst. Des Weiteren wird deutlich, dass der Online-Handel nicht mehr mit dem stationären Handel konkurriert, sondern ihn durch digitale Services bestärkt\footnote{Vgl. \autocite [S.43] {Buss2021}}.
\newline

Die Konsumierenden sollen bei ihrem Einkaufserlebnis freier und flexibler entscheiden können, wie, wann und wo sie einkaufen oder sich informieren möchten. Im Optimalfall sorgt das Unternehmen dafür, dass der Kaufprozess während der gesamten Zeit, einschließlich der Präsentation von Waren, des Verkaufs, des Versands, der Bezahlung und des Kundenservices, auf die persönlichen Bedürfnisse jedes einzelnen Kunden und jeder einzelnen Kundin abgestimmt wird. So soll ein maßgeschneidertes Einkaufserlebnis geboten werden.