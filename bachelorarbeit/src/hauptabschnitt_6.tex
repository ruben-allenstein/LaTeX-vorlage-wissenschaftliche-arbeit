\newpage
\section{Fazit \& Ausblick}\label{hauptabschnitt_6}

Abschließend kann festgestellt werden, dass Omni-Channel für viele Branchen ein wichtiger Trend ist, der es Unternehmen ermöglicht, Konsumierende auf mehreren Kanälen zu erreichen und eine nahtlose und personalisierte Kundenerfahrung zu bieten. Dadurch, dass viele Konsumierende mittlerweile über Smartphone, Tablet, Laptop oder am Computer ihre Einkäufe tätigen oder sich zumindest Informationen zu einem Produkt oder einer Marke einholen, ist die Integration von Online- und Offline-Kanälen entscheidend, um die Kundenbindung zu stärken und den Umsatz zu steigern.
\newline

Die Untersuchungen zeigen, dass die Implementierung von Omni-Channel-Strategien in der Kosmetikbranche noch in den Anfängen steckt und es noch viele Herausforderungen zu bewältigen gibt, wie zum Beispiel die Verwaltung von Bestandsdaten und die Integration von Systemen. Unternehmen, die jedoch erfolgreich eine Omni-Channel-Strategie umsetzen, können eine höhere Kundenbindung und Umsatzsteigerungen erreichen.
\newline

Für zukünftige Forschungsvorhaben könnte es interessant sein, eine tiefere Analyse der Auswirkungen von Omni-Channel auf den Verkauf von Kosmetikprodukten durchzuführen und weitere Faktoren zu berücksichtigen, die die Kundenerfahrung und den Erfolg von Omni-Channel-Strategien beeinflussen. Es könnte auch sinnvoll sein, die Auswirkungen der COVID-19-Pandemie auf die Nutzung von Omni-Channel in der Kosmetikbranche weiter zu untersuchen und mögliche Veränderungen in der Branche zu identifizieren.
\newline

Insgesamt bietet Omni-Channel in der Kosmetikbranche aber viele Chancen, um das Kundenerlebnis zu verbessern und den Erfolg von Unternehmen zu steigern. Unternehmen sollten weiterhin in die Entwicklung und Umsetzung von Omni-Channel-Strategien investieren, um wettbewerbsfähig zu bleiben und auch langfristig erfolgreich zu wirtschaften. Für Unternehmen wird es dabei wichtig sein, auf die Kundenanforderungen einzugehen und sich dabei so auszurichten, dass durch Nutzung neuer Technologien, moderner Softwaresysteme und kreativen Ideen Wettbewerbsvorteile gegenüber der Konkurrenz entstehen können.
\newline

Aus Sicht der stationären Läden wird ein wichtiger Kernpunkt sein, ihre Kundschaft mit einem freundlichen Auftreten und einer ausgewiesenen Expertise bei Produkten zu beraten. Daher ist es wichtig, dass Berater:innen einen Expertenbereich aufbauen, bei dem sie möglichst auf dem neuesten Kenntnisstand sind und ihrer Kundschaft detailliert erklären können, welches Produkt für bestimmte Anforderungen besonders geeignet ist.
\newline

Die Umsetzung einer ganzheitlichen Omni-Channel-Strategie stellt ein Unternehmen vor eine Vielzahl von Herausforderungen. Darunter fallen auch Themen, die aktuell mit Sicherheit noch in den Kinderschuhen stecken. In diesem Zusammenhang steht dem Omni-Channel-Retailing zwar noch eine Wegstrecke bevor, bis es komplett ausgereift ist, aber viele größere Unternehmen in verschiedenen Branchen machen es in Teilen schon vor und sind bereits auf einem sehr guten Weg. Omni-Channel ist ein Geschäftsmodell, das heute bereits Omni-Channel ist ein Geschäftsmodell, das heute bereits bedient und genutzt werden sollte, um in naher Zukunft für den Markt gerüstet zu sein.
\newline

Wenn man die jungen Generationen beobachtet, kann festgestellt werden, wie sich das Kommunikationsverhalten dieser Generationen verändert. Diese Generationen werden diese Trends der Zukunft vorgeben und die Kunden und Kundinnen von morgen sein, die den Unternehmen zeigen, über welche Wege sie kommunizieren wollen. Unternehmen werden sich mittel- und langfristig darauf einstellen müssen, denn wenn das nicht geschieht, besteht die Möglichkeit, den nächsten Klick zu nehmen und ein anderes Zielunternehmen auszuwählen.