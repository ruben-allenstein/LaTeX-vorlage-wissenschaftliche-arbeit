\subsection*{Fragebogen für das Experteninterview}

\subsubsection*{Einleitung}
Vielen Dank, dass Sie sich mit den Fragen im Rahmen meiner Bachelorthesis beschäftigen. Durch die Beantwortung der folgenden Fragen sollen Vor- und Nachteile, aber auch Verbesserungsvorschläge bei der Umsetzung von Omni-Channel-Strategien analysiert werden. Der Fragebogen beschäftigt sich mit der Frage, welche spezifischen Herausforderungen es bei der Implementierung einer Omni-Channel-Strategie im Kosmetikbereich gibt und wie diese speziell von dem Unternehmen BABOR gemeistert bzw. genutzt wurden. Dabei soll herauskristallisiert werden, welche neuen Chancen und Möglichkeiten dabei entstanden sind. Alle Fragen sind freiwillig auszufüllen und sollten Sie bei einer Frage keine Antwort geben können/wollen, kann diese einfach übersprungen werden.

\subsubsection*{Fragen}
\begin{enumerate}
 \item[1.] Bei Omni-Channel-Ansätzen möchte man den Kunden ein möglichst perfektes und rundes Einkaufserlebnis über alle Vertriebskanäle hinweg bieten. Welche Omni-Channel-Konzepte werden bei BABOR eingesetzt?
     \begin{itemize}
            \item{[ ] Click \& Collect}
            \item{[ ] Buy \& Collect}
            \item{[ ] Reserve \& Collect}
            \item{[ ] Weitere Antwortmöglichkeiten:  }
          \end{itemize}
 \item[2.] Welche Vertriebskanäle werden aktuell im B2C genutzt?
     \begin{itemize}
             \item{[ ] Social-Media}
             \item{[ ] Online-Shop}
             \item{[ ] Stationärer Handel}
             \item{[ ] Weitere Antwortmöglichkeiten:  }
           \end{itemize}
 \item[3.] Welche Vertriebskanäle werden aktuell im B2B genutzt?
    \begin{itemize}
              \item{[ ] Social-Media}
              \item{[ ] Online-Shop}
              \item{[ ] Andere Vertriebskanäle}
              \item{[ ] Weitere Antwortmöglichkeiten:  }
            \end{itemize}
 \item[4.] Über welche Berührungspunkte (Touchpoints) kommen die Kunden (B2B als auch B2C) mit dem Unternehmen oder der Marke BABOR in Kontakt?
    \begin{itemize}
               \item{[ ] Broschüre}
               \item{[ ] Print}
               \item{[ ] Pressearbeit}
               \item{[ ] Online-Anzeigen}
               \item{[ ] Social-Media-Werbung}
               \item{[ ] Newsletter}
               \item{[ ] Community / Social-Media}
               \item{[ ] Filialgeschäft}
               \item{[ ] Weitere Antwortmöglichkeiten:  }
             \end{itemize}
 \item[5.] Als Familienunternehmen wurde bei BABOR lange Zeit das B2B Modell in den Mittelpunkt gestellt. Was waren die Gründe für eine Veränderung der Marketing-Strategie zum B2C?
 \item[6.] Was sind die relevantesten Kontaktpunkte mit Kunden, bei denen Informationen über den Kunden gewonnen werden und denen die größte Aufmerksamkeit geschenkt wird?
 \item[7.] BABOR Produkte sind in vielen Kosmetikstudios sehr beliebt. Trotzdem setzt BABOR mittlerweile vermehrt auch auf das B2C-Vertriebsmodell und den Onlinehandel. Welche Prozessveränderungen gab es bei der Umsetzung der neuen Strategieausrichtung?
 \item[8.] Welche Omni-Channel-Ansätze wurden bei dieser Entscheidung berücksichtigt und spielten dabei eine wichtige Rolle?
 \item[9.] Wenn es zu marketingtechnischen Veränderungen kommt, kann dies oftmals auch ein mehrjähriger Prozess sein. Welcher Aufwand war für den Prozess bei BABOR erforderlich, um sich neu auszurichten?
 \item[10.] Welche technischen Prozesse oder Systeme (z.B. neues CRM-System) wurden für die Umsetzung einer Omni-Channel-Strategie grundlegend neu eingeführt und umgesetzt?
 \item[11.] Wenn ein komplett neues CRM-System integriert wird, wird z.B. für die Datenpflege ein sehr hoher Aufwand erwartet. Sind in diesem Zusammenhang bei der Einführung von Systemen irgendwelche (unerwarteten) Probleme entstanden?
 \item[12.] Durch Instagram besteht die Möglichkeit, global eine Reichweite zu generieren. In der Kosmetik-Branche werden auch von Influencern immer wieder Werbung für ein Produkt oder eine Marke veröffentlicht. Welche Vorteile entstehen bei BABOR durch diese Vertriebskanäle und wofür werden diese hauptsächlich genutzt?
    \begin{itemize}
               \item{[ ] Informations- und Anleitungsvideos zu Produkten}
               \item{[ ] Werbung für neue Produkte}
               \item{[ ] Auf die Marke BABOR aufmerksam machen wollen}
               \item{[ ] Weitere Antwortmöglichkeiten:  }
             \end{itemize}
 \item[13.] BABOR Produkte sind bei vielen Kosmetikstudios nach wie vor sehr beliebt. Entsteht durch den Online-Shop eine Zielgruppen-Veränderung oder Zielgruppen-Erweiterung für das Unternehmen? Und was möchte BABOR neben einem guten Produkt tun, um diese Zielgruppen erfolgreich zu erreichen und zufriedenzustellen?
 \item[14.] Über welchen Vertriebskanal wird bei BABOR der größte Umsatz generiert und was wird getan, um das Einkaufserlebnis des Kunden/der Kundin angenehmer zu gestalten?
 \item[15.] Wenn bei einer Parfümerie wie Douglas nach einem BABOR Produkt gesucht wird, gibt es die Möglichkeit, Strategien zu verfolgen, das Einkaufserlebnis für den Kunden oder der Kundin angenehmer zu gestalten? Beispielsweise über einen QR-Code auf dem Produkt, das als Video eine Anleitung zeigt.
 \item[16.] Kann auch von Unternehmen gelernt werden, die in ganz anderen Bereichen/Branchen tätig sind? Wird dabei zum Beispiel auch auf die Big Player wie Amazon geachtet?
 \item[17.] Wie sieht die generelle Marktentwicklung in der Kosmetikbranche und insbesondere bei den Hauptkonkurrenten aus und welche Tendenzen lassen sich für BABOR ableiten?
 \item[18.] Während der Lockdowns aus der Corona-Zeit wurde in vielen Branchen (zum Teil auch notgedrungen) vermehrt auf das Einkaufen im Online-Handel gesetzt. Gab es durch die Zeit der Corona-Pandemie spürbare Veränderungen der Konsumierenden in Bezug auf das Einkaufsverhalten?
 \item[19.] Wenn ja, inwieweit wird versucht, das zu berücksichtigen?
\end{enumerate}