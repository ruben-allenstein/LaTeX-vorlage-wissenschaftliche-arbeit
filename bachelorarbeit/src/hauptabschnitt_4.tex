\newpage
\section{Umsetzung einer Omni-Channel-Strategie bei BABOR}\label{hauptabschnitt_4}

\subsection{Das Unternehmen BABOR}\label{unterabschnitt_4_1}

Die Geschichte des Unternehmens BABOR begann 1956 mit der erfolgreichen Entwicklung einer Rezeptur von dem Chemiker Dr. Michael Babor für das Reinigungsöl HY-ÖL, das bis heute als Verkaufsschlager und Kernprodukt von BABOR auf sämtlichen Plattformen verkauft wird. Das Kosmetik Start-Up wurde im selben Jahr von dem Pharmazeuten Dr. Leo Vossen aufgekauft und gilt seitdem als Familienunternehmen in mittlerweile dritter Generation.
\newline

BABOR ist ein 65 Jahre altes deutsches Kosmetikunternehmen, das Produkte zur Hautpflege im Luxussegment in über 70 Ländern verkauft. Das Lager, die Entwicklung und die Produktion für die eigens entwickelten Produkte finden in Deutschland in der Stadt Aachen statt. Die Marke BABOR umfasst jedoch nicht nur das Reinigungsöl HY-ÖL, sondern auch weitere Produkte im Bereich der Hautpflege. BABOR vermarktet Produkte für Gesichtspflege, Körperpflege, Hand- und Fußpflege, organische Hautpflege, Männerpflege, Sonnenpflege, dekorative Kosmetikprodukte und Düfte.
\newline

BABOR hat sich seit seiner Gründung zu einem Marktführer für professionelle Kosmetik entwickelt. BABOR nutzt vor allem Zutaten aus der Umgebung für die Herstellung ihrer Cremes, Lotionen und Ampullen. Nach eigenen Angaben hat das Unternehmen 2021 insgesamt rund 200 Millionen Euro umgesetzt\footnote{ Vgl. \autocite [online] {Finance2021}}.
\newline

BABOR ist in Deutschland in über 2000 Kosmetikstudios vertreten. Zudem verkaufte BABOR seine Produkte lange Zeit hauptsächlich über den B2B Bereich und pflegt in diesem Bereich viele intensive Kundenbeziehungen mit großen Unternehmen oder Beauty Ketten wie Douglas und Flaconi, wo die Produkte dann schlussendlich in den Regalen stehen und an den Endkunden weiterverkauft werden sollen. Seit 2014 hat BABOR die Unternehmensphilosophie hin zu dem Angebot über mehrere Verkaufskanäle verändert.

\subsection{Forcierung des Onlinegeschäfts}\label{unterabschnitt_4_2}
Durch die Übernahme der dritten Familiengeneration im Management wurde 2016 mit dem früheren L'Oreal Manager Tim Waller zusätzlich auch ein neuer CO-Geschäftsführer eingestellt. Dabei wurde das Ziel gesetzt, mittel- und langfristig auch vermehrt in das Online-Geschäft zu investieren.
\newline

Dafür wurde zunächst zwar hauptsächlich der europäische Markt angepeilt, es wurden aber auch Kooperationen mit ausgewählten US-Spezialisten geschlossen. Im Jahr 2018 wurde die erste nordamerikanische Zusammenarbeit mit der gemeinnützigen Organisation “All Woman Project” bekannt gegeben, die sich dafür einsetzt, Mädchen und Frauen weltweit zu helfen und unabhängig von Alter, Figur oder Rasse positiv und selbstbewusst zu fühlen. BABOR unterstützt die Initiative mit einem limitierten AWPxBABOR Beauty Set und einer kreativen Kampagne\footnote{ Vgl. \autocite [online] {Cision2018}}.
\newline

Über die Online Kanäle soll die Marke BABOR vor allem außerhalb von Mitteleuropa an Bekanntheit gewinnen und das Geschäft weiter vergrößert werden. Bis 2030 verfolgt BABOR das Ziel, von aktuell 30\% auf 70\% der Umsätze aus Regionen außerhalb von Europa zu stoßen und vor allem Umsätze in Asien zu generieren. Die Online-Webstores und Einkäufe über die Online-Marktplätze wie Amazon oder E-Tailern machen aktuell ein Drittel des Umsatzes aus.
\newline

Nachdem bekannt wurde, dass BABOR seit 2016 die Strategieausrichtung verändern wollte, hegten viele Kosmetiker:innen Zweifel daran, langfristig dadurch eigene Kunden und Kundinnen zu verlieren, die die BABOR Produkte zuvor exklusiv im Studio einkauften.
\newline

Kosmetiker:innen bieten BABOR Produkte bereits seit Jahrzehnten in den Kosmetikstudios an und trugen daher einen Teil zur Erfolgsgeschichte bei, dass BABOR in der heutigen Zeit als Premiummarke im Bereich der Kosmetik gehandelt wird. Für den B2B-Bereich sind Kosmetikstudios daher eine tragende Säule für BABOR, weshalb BABOR auch stets das Ziel verfolgte, die Kosmetiker:innen bzw die Kosmetikstudios mit den Hautexpert:innen, mit in diesen Prozess einzubinden, so dass von einer Online Strategie beide Seiten profitieren und wachsen. Dieser Gedanke hat sich ausgezahlt und seit 2014 hat sich der Umsatz des Unternehmens bis 2021 durch die Einbindung des Online-Geschäfts bereits verdoppelt.
Ein weiterer Vorteil des Online-Geschäfts ist, dass man neben dem wachsenden  Bekanntheitsgrad auch sehr nah am Kunden oder an der Kundin sein kann und auf Feedback über Social Media eingehen kann.
\newline

Durch die veränderten Strategieausrichtung, seit 2016 verstärkt auf den Onlinehandel zu setzen, konnte BABOR vergleichsweise gut auf die Pandemie-Jahre reagieren und mit dem Onlinehandel trotz zeitweiser Schließung vieler Kosmetikstudios die Verluste sehr gut wegstecken. So ist der Umsatz im Jahr 2020 nach eigenen Angaben nicht unter den Vorjahresumsatz gefallen, trotz Wegfall vieler Einnahmen der Produkte für die Kosmetikstudios.
\newline

Durch die gute Vorarbeit wurde ein strukturiertes, großes, sehr modernes Lager zur Produktionsabwicklung und Auslieferung am Hauptstandort Aachen aufgebaut. Mit der Ausbruch der Corona-Pandemie und der damit verbundenen Schließung vieler Salons und Geschäfte konnten Kosmetikpartner Produkte weiterhin online anbieten. BABOR bot ihren Geschäftspartnern dabei mit einem Fulfillment Angebot an, den ganzen Logistikprozess zu übernehmen und ihren Partnern dafür eine Provision anzubieten. So konnten beide Seiten profitieren und auch deshalb konnte der Vorjahresumsatz von 2019 trotz der Schließung über mehrere Monate  vieler Geschäfte weiter gehalten werden.
\newline

BABOR verfolgt zudem die Strategie des Consumer Centricity, bei dem nochmal explizit darauf geachtet wird, den Kunden oder die Kundin in den Mittelpunkt zu stellen. Consumer Centricity heißt jedoch nicht, dass selbst nicht auf den Kunden geschaut wird, sondern es wird geschaut, wie der Kunde auf ein Unternehmen schaut. Dadurch, dass der Fokus anders gelegt wird, entstehen zwischen diesen beiden Gedankengängen bei der Analyse verschiedene Ergebnisse zu Verbesserungsmöglichkeiten.

\subsubsection{Entwicklung zum B2C Unternehmen}\label{unterabschnitt_4_2_1}
Ein Grund für die Anpassung der Strategie zu mehr Fokus auf den Onlinehandel waren die veränderten Kundenbedürfnisse. Zwar kaufen Kunden und Kundinnen BABOR Produkte noch immer in den Kosmetikstudios ein, aber mit der Digitalisierung erwarten sie eine höhere Flexibilität und größere Auswahlmöglichkeiten beim Einkaufen. Durch den Marketing Strategiewechsel hin zu einem Omni-Channel-Unternehmen wurde vor allem die Online Präsenz erhöht. Wichtig ist es dabei, dass das Einkaufserlebnis der Konsumierenden gesteigert wird und auch kanalübergreifend Produktinformationen bereitgestellt werden können\footnote{Vgl. \autocite [Online] {Digitalhub2022}}.
\newline

Im Laufe der letzten Jahre folgte die Einbindung eines neuen Webshops. Zudem eröffnete BABOR in attraktiven Gegenden wie Frankfurt oder Hamburg ausgewählte Flagship Stores, in denen sich Konsumierende zum Beispiel vor Ort beraten lassen oder neue Produkte kennenlernen können. Die Flagship Stores dienen dazu, auf die Marke aufmerksam zu machen, Kundennähe zu pflegen und Werbung zu betreiben.
\newline

Zudem wurden Zusammenarbeiten mit ausgewählten Apotheken, bestimmten Parfümerien und Einzelhändlern geschlossen, die ebenfalls BABOR Produkte vertreiben. BABOR Produkte stehen aber mittlerweile auch in Outlets und Department Stores. Allerdings gibt es in vielen dieser Verkaufskanälen keine Produktexpert:innen mehr, daher müssen die Produkte anderweitig in den Regalen für sich sprechen. Für bestimmte Kunden und Kundinnen hat BABOR eigene Teams, mit denen sich unter anderem Key Account Manager beschäftigen. Es wird Wert darauf gelegt, dass die Produkte an einem gut sichtbaren Platz stehen, aber auch Hinweise erscheinen, so dass klar ersichtlich ist, wie das Produkt angewendet wird und was es auszeichnet. Dies kann beispielsweise ein Hinweis mit einem QR-Code, der zum Instagram Social Media Account führt und bei dem ein Beitrag zu einem Anleitungsvideo für das jeweilige Produkt erscheint.

\subsection{Identifizierung von Erfolgsfaktoren und Lücken in den bestehenden Omni-Channel-Strategien}\label{unterabschnitt_4_4}
Omni-Channel-Strategien haben nur nur in der Kosmetikbranche in den letzten Jahren an Bedeutung gewonnen, sondern sind besonders auch für das B2C-Business wichtig, da sie es den Verbrauchern ermöglichen, ein nahtloses Einkaufserlebnis auf verschiedenen Kanälen sicher zu stellen, einschließlich der Online- und Offline-Kanäle. Auch deshalb konzentriert sich BABOR vermehrt auf den B2C-Handel.
Mit der Einführung einer Omni-Channel-Strategie werden die Bedürfnisse und Erwartungen der Kunden und Kundinnen in den Mittelpunkt gestellt. Durch die Sicherstellung eines nahtlosen Übergangs zwischen Online- und Offlinekanälen kann die Zufriedenheit des Kunden sichergestellt werden, indem dabei Aspekte wie Kundenservice, Einkaufsberatung, Cross-Selling, Support, Reparaturangebot etc. integriert werden.
\newline

Auch die Datenanalyse wird für die Umsetzung einer erfolgreichen Omni-Channel-Strategie vorausgesetzt. Damit ist gemeint, dass durch eine umfassende Analyse der Daten ein tiefes Verständnis über die Kundenbedürfnisse und -präferenzen  aufgebaut werden kann. Diese Datenanalyse kann auch dazu beitragen, die Effektivität von Marketing- und Verkaufsaktionen zu messen.
\newline

Die verschiedenen Verkaufskanäle sollten dabei möglichst eine gleiche Qualität beim Einkaufserlebnis bieten, um ein einheitliches Kundenerlebnis zu schaffen. Dabei ist es wichtig, dass das Einkaufserlebnis zum Beispiel mit Hilfe von Datenanalysen personalisiert wird, so dass mit den Kundenbedürfnissen übereinstimmende Produkte angeboten werden können. Die Personalisierung trägt dazu bei, das Engagement und die Loyalität der Kunden zu erhöhen.
\newline

Es kann allerdings auch zu Problemen führen, wenn eine Omni-Channel-Strategie nicht professionell angewendet wird und Fehler gemacht werden. Eine der größten Lücken bei Omni-Channel-Strategien in der Kosmetikbranche ist zum Beispiel die mangelnde Integration zwischen den verschiedenen Verkaufskanälen. Viele Unternehmen haben immer noch Schwierigkeiten, Online- und Offline-Verkaufskanäle nahtlos miteinander zu verknüpfen, was zu einem inkonsistenten Kundenerlebnis führen kann. Ein Beispiel dafür ist, wenn Kunden oder Kundinnen online ein Produkt bestellen und dieses dann im Geschäft abholen möchten, aber Schwierigkeiten auftreten, weil die Bestellungssysteme nicht miteinander verbunden sind. BABOR möchte dieses vermeiden und hat dafür unter anderem ein neues PIM-System eingeführt.
\newline

Ein weiteres Problem ist die unzureichende Analyse der Kundenbedürfnisse. Unternehmen müssen verstehen, welche Kanäle ihre Kunden und Kundinnen bevorzugen und wie sie ihre Produkte am besten anbieten können, um ihre Zielgruppe zu erreichen. Ohne eine gründliche Analyse können Unternehmen ihre Marketing- und Verkaufsaktionen auf die falschen Kanäle ausrichten und somit eine geringere Effektivität erreichen.
\newline

Kunden und Kundinnen bei BABOR nehmen die Erfahrung in den verschiedenen Verkaufskanälen unterschiedlich wahr. Einzelhändler, Online-Shops und direktverkaufende Berater bieten ihren Kunden und Kundinnen unterschiedliche Erfahrungen bieten. Einzelhändler können beispielsweise eine breitere Produktpalette anbieten, während direktverkaufende Berater eine persönlichere Erfahrung und Beratung bieten können.
Zudem bewerten die Konsumierenden die Qualität der Produkte und die Wirksamkeit der Produkte in Bezug auf ihre Bedürfnisse und Erwartungen. Produkte von BABOR werden von vielen Kunden hoch geschätzt, da sie in der Regel hochwertige Inhaltsstoffe und innovative Technologien enthalten.
\newline

Das ist jedoch bei vielen Unternehmen ein großes Problem, denn eine weitere Lücke ist die mangelnde Personalisierung. Kunden und Kundinnen erwarten heute ein personalisiertes Einkaufserlebnis, bei dem Produkte und Empfehlungen auf ihre Bedürfnisse und Präferenzen zugeschnitten sind. Ohne eine umfassende Datenerfassung und Analyse können Unternehmen jedoch Schwierigkeiten haben, personalisierte Angebote bereitzustellen, was wiederum zu einer geringeren Kundenbindung führen kann.
\newline

Zudem fehlt es vielen Unternehmen an Innovation, um die Omni-Channel-Strategien zu verbessern. Neue Technologien wie Augmented Reality, virtuelle Try-on oder personalisierte Beauty-Beratung werden nur von wenigen Marken angeboten, was die Kundenerfahrung weiter verbessern könnte. Wenn Unternehmen nicht innovativ sind, können sie es schwer haben, sich von ihren Mitbewerbern abzuheben und die Erwartungen ihrer Kunden zu erfüllen. BABOR möchte sein Online-Hautanalyse-Tool in den nächsten fünf Jahren bis 2028 über Künstliche Intelligenz auswerten lassen.
\newline

Länder in Asien wie China und Südkorea haben in den letzten Jahren bereits eine erfolgreiche Entwicklung im Bereich des Omni-Channeling hingelegt, wobei insbesondere der Einzelhandel in China ein Vorreiter ist. Daher lohnt es sich für Unternehmer aus Europa, auch einen Blick auf den Asien-Pazifik-Raum zu werfen und bestimmte Punkte zu adaptieren, wie die Unternehmen sich dort in den vergangenen zehn Jahren entwickelt haben. Auch weil die Menschen in Asien beim Thema Datensicherheit offener sind als Menschen aus Europa, wird dort im Handel viel Wert auf bargeldlose und elektronische Bezahlmethoden gelegt. In China sind viele Einzelhändler bereits seit einigen Jahren im Bereich des Omni-Channel-Handels tätig und bieten ihren Kunden ein nahtloses Einkaufserlebnis über verschiedene Kanäle wie Online-Plattformen, mobile Apps und stationäre Geschäfte. Ein Beispiel dafür ist die chinesische Plattform Alibaba, die sowohl Online- als auch Offline-Verkaufskanäle bietet, einschließlich einer breiten Palette von Einzelhändlern und Marken, die ihre Produkte über die Plattform verkaufen.
\newline

In Europa haben viele Einzelhändler auch in den letzten Jahren in den Bereich des Omni-Channel-Handels investiert, jedoch sind die Fortschritte nicht so schnell wie in Asien. Es gibt jedoch einige Unternehmen, die im Bereich des Omni-Channel-Handels besonders fortschrittlich sind, wie zum Beispiel Zalando, ein deutsches Online-Modegeschäft, das auch physische Geschäfte betreibt und ein nahtloses Einkaufserlebnis über verschiedene Kanäle bietet. Auch die Verfügbarkeit von Produkten in verschiedenen Kanälen kann sich auf die Wahrnehmung von Kunden auswirken. Wenn Produkte schwer zu finden sind oder oft ausverkauft sind, kann dies Kunden enttäuschen und ihre Entscheidungen beeinflussen. BABOR verfolgt einen Plan, dass Kunden und Kundinnen Produkte schnell, einfach und ansprechend finden und einfach kaufen können.
\subsection{Beispielhafter Verlauf zur Einführung eines neuen PIM-Systems}\label{unterabschnitt_4_3}
Das 2021 neu eingeführte Akeneo \ac{pim}-System bei BABOR dient als neue Produktdaten-Plattform. Vor der Umsetzung stand BABOR vor der besonderen Herausforderung, dass mit der Einführung eines zentralen Systems zur Verwaltung von Produktinformationen Datensilos aufgebrochen und konsistente, abteilungsübergreifende Prozesse zur Datenpflege etabliert werden sollten. Noch bis Dezember 2020 wurden Produktinformationen bei BABOR dezentral und unabhängig von jeder Abteilung gepflegt. Dafür wurde mit allen beteiligten Abteilungen ein globaler Prozess  erarbeitet.

Analog dazu mussten auch die Inhalte, die Attribute und Merkmale der eigentlichen Produkte zwischen den Abteilungen abgestimmt und definiert werden. Dabei wurde unter Moderation von Vanilla Reply ein gemeinsames Verständnis hinsichtlich des Datenmodells erarbeitet.

Vor dem eigentlichen Projektstart wurde in einer Evaluationsphase bestätigt, dass sich die ermittelten Anforderungen an das Produktinformationsmanagement mit der Lösung Akeneo umsetzen lassen. Durch das neue Backend System werden eine Sicherung der Wettbewerbsfähigkeit sowie schlankere und effizientere Arbeitsprozesse eingeschlagen.
\newline

BABOR wollte ein zentrales System für interne Daten, damit alle Abteilungen des Unternehmens die Daten auf gleiche Art und Weise ablegen und speichern können und wissen, wie sie auf die benötigten Daten zugreifen können. Für den Start der Umsetzung fanden zunächst Workshops und Prozessberatungen mit den einzelnen Abteilungen statt. Dabei wurde herausgearbeitet, wie die jeweilige Abteilung ihre Daten speichert, ablegt und pflegt. Das Ergebnis der Workshops war die Erkenntnis, dass die Abteilungen bisher keine integrierten Prozesse haben und verteilt gespeicherte Dateien für die Verwaltung der Produktdaten verwenden.
\newline

Dabei wurde einerseits betrachtet, welche Daten in der jeweiligen Abteilung von außen benötigt werden, um ihre Arbeitsschritte durchführen zu können, andererseits aber auch die Daten zu erfassen, die aus der jeweiligen Abteilung heraus für das gesamte Unternehmen und die Vertriebskanäle bereitgestellt werden können.
\newline

Die Erarbeitung und Umsetzung eines Rollen- und Rechtesystems war eine weitere wichtige Anforderung. Jede Abteilung sollte genau die für den jeweiligen Workflowschritt und nach Zuständigkeit benötigten Rechte und Zugriffsmöglichkeiten zur Pflege und Kontrolle der Produktdaten bekommen.
\newline

Auch das Thema Übersetzungen war Teil des neuen Backend Systems. Bisher wurde die Übersetzung der Produkttexte von neuen Produkten direkt im E-Commerce-Shop - und damit wieder dezentral - vorgenommen, jedoch ermöglichte das intern keine gute Übersicht über den Stand und die Qualität dieser Inhalte. Die Anforderung an die Vanilla Reply GmbH war daher, die Hauptsprachen Englisch, Deutsch, Französisch und Niederländisch in das zentrale Akeneo PIM zu integrieren. Teilweise unterliegen Produkte in einigen Ländern aufgrund ihrer Inhaltsstoffe oder Wirkweise besonderen Anforderungen. Babor hat durch das neue System mehr Kontrolle über derartige Produkte und Produktinformationen und kann die rechtlichen Rahmenbedingungen besser prüfen und aussteuern.
\newline

Für die Einführung einer neuen Datenzentrale wurde das PIM-System Akeneo verwendet, um die internen Prozesse zu modernisieren und ein verlässliches System für die Bereitstellung von Produktinformationen aufzubauen. In dem System werden alle Daten zusammen abgelegt und gespeichert sowie bei Bedarf und je nach Art der Verwendung übersetzt und/oder vertriebskanalspezifisch angereichert.
\newline

In einer ausführlichen Konzeptphase wurde zunächst der Ist-Prozess analysiert und dokumentiert. Daraus geht hervor,  wie die Datenablage, -speicherung und -pflege vorher organisiert war und welche (technischen und/oder organisatorischen) Schnittstellen zwischen den beteiligten Systemen bestehen. Auf dieser Basis wurden Optimierungen hinsichtlich eines konsistenten Workflows sowie der Schaffung der richtigen Schnittstellen konzipiert.
\newline

In mehreren Workshops wurden die Anforderungen und Möglichkeiten der einzelnen Abteilungen ermittelt. Gemeinsam wurde ein passender Prozess erarbeitet, auf dem das Rechte- und Rollenmanagement abgestimmt wurde.
Dies erfolgte Hand in Hand mit der Entwicklung des auf die Produkte von Babor optimierten Datenmodells.
\newline

Auch das Thema Übersetzungen wurde mit Akeneo für die ersten vier Hauptsprachen hervorragend umgesetzt und zentralisiert. Englisch, Deutsch, Niederländisch und Französisch sind schon in der zentralen Verwaltung hinzugekommen. Für zusätzliche Märkte kann der Datenstand im Akeneo PIM unkompliziert um weitere Sprachen ergänzt werden. Bei der Einführung eines weiteren Marktes und einer damit ggf. notwendigen neuen Sprache leitet das Akeneo PIM die Benutzer:innen intuitiv an, die fehlenden Informationen (Übersetzungen) zu pflegen. Das Akeneo PIM-System unterstützt BABOR zudem auch bei der fristgerechten Bearbeitung von Produktlaunches.
\newline

In der umgesetzten Phase des Projektes wurde das E-Commerce-System als primärer Vertriebskanal an die neu implementierten Prozesse angeschlossen. Für den Online-Verkauf werden spezifische und wertvolle Informationen daher wiederum von den BABOR E-Commerce Experten im Akeneo PIM System ergänzt und die Produktinformationen so für diesen Vertriebskanal optimiert.
\newline

BABOR hat dadurch ein System, das die interne Zusammenarbeit der Abteilungen und des gesamten Unternehmens modernisiert, weil alle Produktinformationen in einem System zentralisiert wurden. Jede Abteilung kann ihre Daten im System einsehen, pflegen und für andere bereitstellen. Außerdem sind die Daten von BABOR im PIM-System von Akeneo markt- und kanalspezifisch verfügbar.
\newline

Der Nutzen für BABOR ist groß, weil die gesamte interne abteilungsübergreifende Arbeitsweise von BABOR modernisiert wurde. Alle Produktinformationen liegen durch die Integrierung des PIM-Systems zentralisiert in einem System und alle Abteilungen können dort die für sie relevanten Daten sehen, pflegen und für andere Abteilungen und Systeme zur Weiterverarbeitung bereitstellen. Die internen Prozesse sind mit der Einführung des Systems deutlich schlanker geworden und die Mitarbeiter von BABOR können effizienter arbeiten als vorher.


\subsection{Analyse von Online- und Offline-Präsenz, Angeboten, Kundenservice und Marketingaktivitäten}\label{unterabschnitt_4_5}
BABOR hat eine gut gestaltete Website, die es den Kunden ermöglicht, Produkte online zu kaufen und sich über die neuesten Angebote und Aktionen zu informieren. Die Website ist benutzerfreundlich und ansprechend gestaltet, was das Einkaufserlebnis für den Kunden verbessert. Das Unternehmen ist auch auf Social-Media-Plattformen wie Facebook, Instagram und YouTube aktiv, um seine Produkte zu bewerben und sich mit den Kunden zu engagieren. Durch regelmäßige Beiträge und Interaktion mit Followern, fördert BABOR ein starkes Markenimage.
\newline

BABOR ist in verschiedenen Ländern präsent und hat zahlreiche Einzelhandelsgeschäfte, in denen Kunden die Produkte vor Ort testen und kaufen können. Die Geschäfte sind ansprechend gestaltet und bieten ein angenehmes Einkaufserlebnis. Darüber hinaus arbeitet BABOR mit ausgewählten Spas und Salons zusammen, um seine Produkte zu vertreiben und spezielle Behandlungen anzubieten. Dies ermöglicht es dem Unternehmen, sein Angebot zu erweitern und die Marke in der Wellnessbranche zu positionieren. Auch in einzelnen Douglas-Läden, bei denen BABOR Produkte zur Verfügung stehen, werden QR-Codes angeboten, die wiederum weitere Produktinformationen bieten.
\newline

Soziale Medien-Plattformen nutzen Algorithmen, um Inhalte anzuzeigen, die den Benutzer:innen am besten gefallen oder am wahrscheinlichsten ihr Interesse wecken. Dies bedeutet, dass Unternehmen wie BABOR möglicherweise nicht in der Lage sind, ihre Zielgruppe auf sozialen Medien effektiv zu erreichen, wenn die Algorithmen sie nicht priorisieren.
BABOR hat so nur eine begrenzte Kontrolle über soziale Medien-Plattformen und kann nicht garantieren, dass die Inhalte oder Werbung auf der jeweiligen Plattform wie Instagram, Facebook oder YouTube erscheinen werden. Darüber hinaus können Änderungen an den Plattformrichtlinien oder -algorithmen die Sichtbarkeit von Inhalten und Werbung beeinträchtigen. Zudem gibt es auf diesen Plattformen auch keine integrierten E-Commerce-Funktionen, weshalb es schwierig ist, den Erfolg von Kampagnen auf diesen Plattformen zu messen.
\newline

Obwohl BABOR bereits eine starke Online- und Offline-Präsenz hat, gibt es immer Möglichkeiten zur Verbesserung. BABOR könnte personalisierte Marketingstrategien und Angebote einführen, um das Einkaufserlebnis für Kunden zu verbessern. Zum Beispiel könnte das Unternehmen personalisierte Hautpflege Empfehlungen basierend auf den individuellen Bedürfnissen des Kunden anbieten.
Zudem könnte BABOR das Feedback der Kunden sammeln und darauf reagieren, um das Kundenerlebnis weiter zu verbessern. Dies kann zum Beispiel durch regelmäßige Umfragen geschehen, um zu erfahren, was Kunden von den Produkten und dem Einkaufserlebnis halten.
BABOR könnte sein Angebot an Produkten und Dienstleistungen weiter ausbauen, indem das Unternehmen beispielsweise seine Produktlinien erweitert oder neue Behandlungen in Zusammenarbeit mit den Instituten und Kosmetikstudios einführt.
Es empfiehlt sich außerdem,  das Einkaufserlebnis für Kunden durch personalisierte Beratung oder exklusive Angebote zu verbessern. Zum Beispiel könnte BABOR seinen VIP-Kunden exklusive Behandlungen oder Rabatte anbieten.
\newline

BABORs Kundenservice sticht bei den Kunden und Kundinnen hervor. Das Unternehmen bietet seinen Kunden und Kundinnen eine Vielzahl von Support-Optionen an, darunter eine Service-Hotline und E-Mail-Unterstützung. BABORs Kundenservice ist auch auf den sozialen Medien aktiv und bietet Kunden und Kundinnen die Möglichkeit, sich mit dem Unternehmen in Verbindung zu setzen und Fragen oder Anliegen zu teilen.
\newline

Ein weiterer wichtiger Aspekt des Kundenservice von BABOR ist das Engagement für Schulungen und Fortbildungen von Fachkräften. Das Unternehmen bietet Schulungen und Workshops für Kosmetiker und andere Fachleute an, um sicherzustellen, dass sie die Produkte und Behandlungen von BABOR richtig anwenden und ihren Kunden das bestmögliche Erlebnis bieten können.
